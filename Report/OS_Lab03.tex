\documentclass[12pt]{beamer}
\title{OSLAB Project 03}
\author{YaoShunyu  LvYingzhe}
\institute{10152130255 LvYingzhe}
\usepackage{beamerthemesplit}
\usepackage{listings}
\usepackage{graphicx}
\usetheme{Pittsburgh}
\usecolortheme{seagull}
\usepackage{courier}

\lstset{basicstyle=\ttfamily}

\begin{document}
\frame{\titlepage}

\section{Design}
\frame{
\frametitle{Basic Conceptions}
\begin{itemize}
\item Maintain an ordered free list
\item De-fragmentation after free
\end{itemize}
}

\begin{frame}[fragile]
\frametitle{Metadata}
\begin{lstlisting}[frame=single]
typedef struct _Node {
    struct _Node *next;
    int size;
} Node;

typedef struct _Header {
    int size;
    struct _Header *magic;
} Header;
\end{lstlisting}
\end{frame}

\subsection{Allocation}
\frame{
\frametitle{How Allocation Works?}
\begin{itemize}
\item Verify argument
\item Round up with bitwise operation
\item Find a suitable block
\item Split/Remove from free list
\item Transform into Header and return
\end{itemize}
}

\frame{
\frametitle{Split or Remove?}
\begin{center}
\begin{tabular}{c|c}
\hline
Condition& Action\\
\hline
pNode$\rightarrow$size $\geq$ size + sizeof(Node)& Split\\
pNode$\rightarrow$size $<$ size + sizeof(Node)& Remove\\
\hline
\end{tabular}
\end{center}
%\centering{sizeof(Node) = 16 bytes}
}

\subsection{Free}
\frame{
\frametitle{How Free Works?}
\begin{itemize}
\item Verify pointer argument
\item Transform into Node
\item Insert into free list
\item Try to merge with left/right neighbour
\end{itemize}
}

\subsection{Drawback}
\frame{
\frametitle{Dilemma}
16-byte header can store mere information!\\
\vspace{2em}
Free list? Fully connected? We want both!
}

\section{Another Implementation}
\frame{
\frametitle{Design}
\begin{itemize}
\item 24-byte Header
\item Maintain a free list
\item Link between neighbours
\end{itemize}
}

\begin{frame}[fragile]
\frametitle{24-byte Header}
\begin{lstlisting}
typedef struct _Node {
    struct _Node *free_next;
    struct _Node *pre;
    int size;
} Node;
\end{lstlisting}
\end{frame}

\frame{
\frametitle{Improvement}
\begin{itemize}
\item Same allocation speed
\item Much faster free speed
\end{itemize}
}

\section{Performance}
\subsection{Allocation}
\begin{frame}[fragile]
\frametitle{Allocation without Free}
\begin{lstlisting}
for i <- 1 to N:
	ptr[i] = alloc(rand(1, BOUND), STYLE)
\end{lstlisting}
\end{frame}

\frame{
\begin{center}
\includegraphics[height=18em]{Figure_4.png}
\end{center}
}

\frame{
\frametitle{Conclusion}
\large
The allocation is not the bottleneck.
}

\subsection{Short Free List}
\begin{frame}[fragile]
\frametitle{Random Allocation/Free Calls}
\begin{lstlisting}
op <- rand(ALLOC, FREE)
if op = ALLOC:
	size <- rand(1, BOUND)
	list.add(alloc(size))
else if op = FREE and list.size() > 0:
	index <- rand(list.size())
	free(list[index])
	list.remove(index)
\end{lstlisting}
\end{frame}

\frame{
\begin{center}
\includegraphics[height=18em]{Figure_1.png}
\end{center}
}

\frame{
\frametitle{Conclusion}
\large{
Both 16-byte and 24-byte stand the violent test,\\
despite that system performs better.
}
}

\subsection{Allocation Styles}
\frame{
\frametitle{Different Allocation Styles}
\large{
Same test method as short free list\\
\vspace{2em}
Change allocation styles respectively
}
}

\frame{
\begin{center}
\includegraphics[height=18em]{Figure_5.png}
\end{center}
}

\frame{
\begin{center}
\includegraphics[height=18em]{Figure_6.png}
\end{center}
}

\frame{
\frametitle{Conclusion}
\large{
Performance:\\
First Fit $>$ Best Fit $>$ Worst Fit
}
}

\subsection{Long Free List}
\begin{frame}[fragile]
\frametitle{Numerous Free after Allocation}
\begin{lstlisting}
indexes <- Fisher_Yates(N)
for i <- 1 to N:
	ptr[i] = alloc(rand(1, BOUND))
for i <- 1 to N:
	free(ptr[indexes[i]])
\end{lstlisting}
\end{frame}

\frame{
\begin{center}
\includegraphics[height=18em]{Figure_2.png}
\end{center}
}

\frame{
\begin{center}
\includegraphics[height=18em]{Figure_3.png}
\end{center}
}

\frame{
\frametitle{Conclusion}
\large
When free list is long,\\
the cost of free for 16-byte version is significant,\\
while 24-byte version performs like system malloc.
}

\section{Summary}
\newcommand{\tabincell}[2]{\begin{tabular}{@{}#1@{}}#2\end{tabular}}
\frame{
\frametitle{Comparison}
\begin{center}
\begin{tabular}{c|c|c|c}
\hline
Method& Speed& Fragment& Application\\
\hline
\tabincell{c}{Dynamic\\Allocator}& Moderate& \tabincell{c}{Little External\\No Internal}& malloc\\
\hline
\tabincell{c}{Buddy\\Allocator}& Fast& \tabincell{c}{Little External\\Much Internal}& jelloc\\
\hline
\tabincell{c}{Slab\\Allocator}& Very Fast& \tabincell{c}{No External\\Much Internal}& kalloc\\
\hline
\end{tabular}
\end{center}
}

\frame{
\frametitle{Summary}
\large{
Balance the cost and gain is the eternal theme.\\
\vspace{2em}
Nothing is best, only most suitable.
}
}

\frame{
\frametitle{Fin}
\Large{Thank you!}
}

\frame{
\begin{center}
\begin{tabular}{c|c}
\hline
Condition& Action\\
\hline
pNode$\rightarrow$size $\geq$ size+sizeof(Node)& Split\\
pNode$\rightarrow$size $==$ size + 8& Remove with extra 8-byte\\
pNode$\rightarrow$size $==$ size& Remove\\
\hline
\end{tabular}
\end{center}
}

\frame{
%\frametitle{Example 1}
\begin{center}
\includegraphics[height=16em]{P1.png}
\end{center}
}

\frame{
%\frametitle{Example 2}
\begin{center}
\includegraphics[height=16em]{P2.png}
\end{center}
}

\frame{
%\frametitle{Example 3}
\begin{center}
\includegraphics[height=16em]{P3.png}
\end{center}
}

\end{document}